\documentclass[]{article}
\usepackage{lmodern}
\usepackage{amssymb,amsmath}
\usepackage{ifxetex,ifluatex}
\usepackage{fixltx2e} % provides \textsubscript
\ifnum 0\ifxetex 1\fi\ifluatex 1\fi=0 % if pdftex
  \usepackage[T1]{fontenc}
  \usepackage[utf8]{inputenc}
\else % if luatex or xelatex
  \ifxetex
    \usepackage{mathspec}
  \else
    \usepackage{fontspec}
  \fi
  \defaultfontfeatures{Ligatures=TeX,Scale=MatchLowercase}
\fi
% use upquote if available, for straight quotes in verbatim environments
\IfFileExists{upquote.sty}{\usepackage{upquote}}{}
% use microtype if available
\IfFileExists{microtype.sty}{%
\usepackage{microtype}
\UseMicrotypeSet[protrusion]{basicmath} % disable protrusion for tt fonts
}{}
\usepackage[margin=1in]{geometry}
\usepackage{hyperref}
\hypersetup{unicode=true,
            pdftitle={Signaling.R},
            pdfauthor={mattk},
            pdfborder={0 0 0},
            breaklinks=true}
\urlstyle{same}  % don't use monospace font for urls
\usepackage{color}
\usepackage{fancyvrb}
\newcommand{\VerbBar}{|}
\newcommand{\VERB}{\Verb[commandchars=\\\{\}]}
\DefineVerbatimEnvironment{Highlighting}{Verbatim}{commandchars=\\\{\}}
% Add ',fontsize=\small' for more characters per line
\usepackage{framed}
\definecolor{shadecolor}{RGB}{248,248,248}
\newenvironment{Shaded}{\begin{snugshade}}{\end{snugshade}}
\newcommand{\KeywordTok}[1]{\textcolor[rgb]{0.13,0.29,0.53}{\textbf{#1}}}
\newcommand{\DataTypeTok}[1]{\textcolor[rgb]{0.13,0.29,0.53}{#1}}
\newcommand{\DecValTok}[1]{\textcolor[rgb]{0.00,0.00,0.81}{#1}}
\newcommand{\BaseNTok}[1]{\textcolor[rgb]{0.00,0.00,0.81}{#1}}
\newcommand{\FloatTok}[1]{\textcolor[rgb]{0.00,0.00,0.81}{#1}}
\newcommand{\ConstantTok}[1]{\textcolor[rgb]{0.00,0.00,0.00}{#1}}
\newcommand{\CharTok}[1]{\textcolor[rgb]{0.31,0.60,0.02}{#1}}
\newcommand{\SpecialCharTok}[1]{\textcolor[rgb]{0.00,0.00,0.00}{#1}}
\newcommand{\StringTok}[1]{\textcolor[rgb]{0.31,0.60,0.02}{#1}}
\newcommand{\VerbatimStringTok}[1]{\textcolor[rgb]{0.31,0.60,0.02}{#1}}
\newcommand{\SpecialStringTok}[1]{\textcolor[rgb]{0.31,0.60,0.02}{#1}}
\newcommand{\ImportTok}[1]{#1}
\newcommand{\CommentTok}[1]{\textcolor[rgb]{0.56,0.35,0.01}{\textit{#1}}}
\newcommand{\DocumentationTok}[1]{\textcolor[rgb]{0.56,0.35,0.01}{\textbf{\textit{#1}}}}
\newcommand{\AnnotationTok}[1]{\textcolor[rgb]{0.56,0.35,0.01}{\textbf{\textit{#1}}}}
\newcommand{\CommentVarTok}[1]{\textcolor[rgb]{0.56,0.35,0.01}{\textbf{\textit{#1}}}}
\newcommand{\OtherTok}[1]{\textcolor[rgb]{0.56,0.35,0.01}{#1}}
\newcommand{\FunctionTok}[1]{\textcolor[rgb]{0.00,0.00,0.00}{#1}}
\newcommand{\VariableTok}[1]{\textcolor[rgb]{0.00,0.00,0.00}{#1}}
\newcommand{\ControlFlowTok}[1]{\textcolor[rgb]{0.13,0.29,0.53}{\textbf{#1}}}
\newcommand{\OperatorTok}[1]{\textcolor[rgb]{0.81,0.36,0.00}{\textbf{#1}}}
\newcommand{\BuiltInTok}[1]{#1}
\newcommand{\ExtensionTok}[1]{#1}
\newcommand{\PreprocessorTok}[1]{\textcolor[rgb]{0.56,0.35,0.01}{\textit{#1}}}
\newcommand{\AttributeTok}[1]{\textcolor[rgb]{0.77,0.63,0.00}{#1}}
\newcommand{\RegionMarkerTok}[1]{#1}
\newcommand{\InformationTok}[1]{\textcolor[rgb]{0.56,0.35,0.01}{\textbf{\textit{#1}}}}
\newcommand{\WarningTok}[1]{\textcolor[rgb]{0.56,0.35,0.01}{\textbf{\textit{#1}}}}
\newcommand{\AlertTok}[1]{\textcolor[rgb]{0.94,0.16,0.16}{#1}}
\newcommand{\ErrorTok}[1]{\textcolor[rgb]{0.64,0.00,0.00}{\textbf{#1}}}
\newcommand{\NormalTok}[1]{#1}
\usepackage{graphicx,grffile}
\makeatletter
\def\maxwidth{\ifdim\Gin@nat@width>\linewidth\linewidth\else\Gin@nat@width\fi}
\def\maxheight{\ifdim\Gin@nat@height>\textheight\textheight\else\Gin@nat@height\fi}
\makeatother
% Scale images if necessary, so that they will not overflow the page
% margins by default, and it is still possible to overwrite the defaults
% using explicit options in \includegraphics[width, height, ...]{}
\setkeys{Gin}{width=\maxwidth,height=\maxheight,keepaspectratio}
\IfFileExists{parskip.sty}{%
\usepackage{parskip}
}{% else
\setlength{\parindent}{0pt}
\setlength{\parskip}{6pt plus 2pt minus 1pt}
}
\setlength{\emergencystretch}{3em}  % prevent overfull lines
\providecommand{\tightlist}{%
  \setlength{\itemsep}{0pt}\setlength{\parskip}{0pt}}
\setcounter{secnumdepth}{0}
% Redefines (sub)paragraphs to behave more like sections
\ifx\paragraph\undefined\else
\let\oldparagraph\paragraph
\renewcommand{\paragraph}[1]{\oldparagraph{#1}\mbox{}}
\fi
\ifx\subparagraph\undefined\else
\let\oldsubparagraph\subparagraph
\renewcommand{\subparagraph}[1]{\oldsubparagraph{#1}\mbox{}}
\fi

%%% Use protect on footnotes to avoid problems with footnotes in titles
\let\rmarkdownfootnote\footnote%
\def\footnote{\protect\rmarkdownfootnote}

%%% Change title format to be more compact
\usepackage{titling}

% Create subtitle command for use in maketitle
\newcommand{\subtitle}[1]{
  \posttitle{
    \begin{center}\large#1\end{center}
    }
}

\setlength{\droptitle}{-2em}

  \title{Signaling.R}
    \pretitle{\vspace{\droptitle}\centering\huge}
  \posttitle{\par}
    \author{mattk}
    \preauthor{\centering\large\emph}
  \postauthor{\par}
      \predate{\centering\large\emph}
  \postdate{\par}
    \date{Fri Oct 19 16:31:34 2018}


\begin{document}
\maketitle

Compute Cell-cell interaction probability

We compute three expressions. Here is a sample:
\eqn{K_{i,j} = \frac{\alpha_{i,j}}{\alpha_{i,j} + \beta_{i,j}}}

@param M a matrix of expression values for each cell (rows) and gene
(columns) @param cluster\_labels a vector of cluster labels @param H a
nonnegative matrix such that W = H*t(H), H\_\{i,j\} is the cells\_weight
by which cell i belongs to the jth cluster @param
gene\_expression\_threshold for n cells, for
\code{gene_expression_threshold} = m, dont consider genes expressed in
more than n-m cells or genes expressed in less than m cells @param
n\_features number of marker genes per cluster to retrieve

@return a table of marker genes

\begin{Shaded}
\begin{Highlighting}[]
\NormalTok{GetSignalingPartners <-}\StringTok{ }\ControlFlowTok{function}\NormalTok{(counts_data,}
\NormalTok{                           cluster_labels,}
\NormalTok{                           H,}
\NormalTok{                           gene_expression_threshold,}
\NormalTok{                           n_features)\{}
  \CommentTok{# filter the data according to desired samples (cells) and features (genes)}

\NormalTok{  M <-}\StringTok{ }\KeywordTok{SelectData}\NormalTok{(counts_data,}
\NormalTok{                  gene_expression_threshold,}
\NormalTok{                  n_features)}

  \CommentTok{# normalize the expression cell by cell}
\NormalTok{  M}\OperatorTok{$}\NormalTok{M_variable <-}\StringTok{ }\KeywordTok{apply}\NormalTok{(M}\OperatorTok{$}\NormalTok{M_variable, }\DecValTok{2}\NormalTok{, }\ControlFlowTok{function}\NormalTok{(x)\{}
\NormalTok{    x }\OperatorTok{/}\StringTok{ }\KeywordTok{sum}\NormalTok{(x)}
\NormalTok{  \})}

  \CommentTok{# for each gene (rows), for each cluster (columns)}
  \CommentTok{# get cell-weighted expression score}
\NormalTok{  gene_cluster_scores <-}\StringTok{ }\NormalTok{M}\OperatorTok{$}\NormalTok{M_variable }\OperatorTok\StringTok{ }\NormalTok{H}

  \CommentTok{# select max cluster score}
\NormalTok{  gene_assignments <-}\StringTok{ }\KeywordTok{apply}\NormalTok{(gene_cluster_scores, }\DecValTok{1}\NormalTok{, }\ControlFlowTok{function}\NormalTok{(x)\{}
    \KeywordTok{c}\NormalTok{(}\KeywordTok{which}\NormalTok{(x }\OperatorTok{==}\StringTok{ }\KeywordTok{max}\NormalTok{(x)), }\KeywordTok{max}\NormalTok{(x))}
\NormalTok{  \})}


\NormalTok{  marker_table <-}\StringTok{ }\KeywordTok{cbind}\NormalTok{(M}\OperatorTok{$}\NormalTok{gene_use, }\KeywordTok{t}\NormalTok{(gene_assignments))}
  \KeywordTok{return}\NormalTok{(marker_table)}
\NormalTok{\}}
\end{Highlighting}
\end{Shaded}


\end{document}
